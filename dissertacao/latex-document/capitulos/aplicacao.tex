\chapter{Clusterização como pré-processamento na compressão de textos}
Em uma sociedade cada vez informatizada, a quantidade de dados textuais produzidas é cada vez maior e consequentemente a demanda para o armazenamento de grandes volumes de dados cresce.
Apesar dos avanços de hardware para armazenamento que tivemos nas últimas décadas (hardwares cada vez menores com maior capacidade de armazenamento),
a \emph{compressão de dados} ainda se mostra uma arma poderosa para aumentar a capacidade de armazenamento de informação com menor custo.

Nos capítulos anteriores, construímos a estrutura teórica necessária para compreender a \emph{compressão de texto} e sua relação com a \emph{teoria da informação}.
Fica claro que os algoritmos apresentados, tomam vantagem de alguma redundância presente na mensagem para representar a informação de maneira mais eficiente.
Isto é, a \textbf{eficiência} destes algoritmos está intimamente ligada com a \textbf{entropia} do texto a ser comprimido,
 nos dando um indício de que \textbf{minimizar} a entropia, pode ser uma forma de \textbf{maximizar} a \textbf{taxa de compressão}.
 
O capítulo~\ref{cap:clus} apresenta uma técnica para organizar o texto em \emph{grupos de mensagens similares} (\textbf{clusterização}), o que significa diminuir a variação de termos dentro de cada cluster (ou reduzir a sua \textbf{entropia}).
Neste capítulo, exploraremos de maneira empírica a relação entre a entropia associada a um texto e a taxa de compressão dos algoritmos apresentados no capítulo~\ref{cap:comp}.

A implementação descrita a seguir utiliza a \textbf{clusterização} como pré-processamento de uma base de dados textual, visando uma melhora na performance dos algoritmos de compressão \emph{sem perda}.
Foram realizados experimentos utilizando ou não o pré-processamento, afim de medir o impacto do processo nos diversos tipos de algoritmos testados (que serão listados em seções posteriores) comprando as diferentes perfomances obtidas.

\pagebreak

\section{Base de dados analisada}
tecnologia utilizada
descrever a base de dados utilizada, e o porque.
Descrever as características gerais e grupos iniciais;
pre tratamento das informações;

\section{Compressão de textos}
Tecnologias / linguagem utilizada
Detalhes da implementacao, diagrama de classes etc
Incluir código

\section{Partição de dados}
\subsection{Partição aleatória}
explicar o racional por traz da particao de dados
\subsection{Clusterizacao}
Tecnologia e implementação
Mostrar parametros
mostrar gráficos

\section{Resultados}
\subsection{Metodologia}
Explicar sobre a extassao de medidas de comparacao

\subsection{Resultados pré clusterizacao}

\subsection{Resultados pós clusterizacao}

\section{Conclusao}

\section{Trabalhos relacionados e melhorias}




