% ---------------------------------------------------------------------------
% ---------------------------------------------------------------------------
% Junção de templates encontrados no Overleaf
% facilitando a vida do aluno de pós da UFABC
% Modelo LaTex para preparação do documento final de Dissertação de Mestrado
% ninguém do programa de pós da informação validou se presta
% ---------------------------------------------------------------------------
% ---------------------------------------------------------------------------

\documentclass[
	% -- opções da classe memoir --
	12pt,					% tamanho da fonte
	openright,				% capítulos começam em pág ímpar (insere página vazia caso preciso)
	twoside,					% para impressão em verso e anverso. Oposto a oneside
	a4paper,					% tamanho do papel. 
	% -- opções da classe abntex2 --
	%chapter=TITLE,			% títulos de capítulos convertidos em letras maiúsculas
	%section=TITLE,			% títulos de seções convertidos em letras maiúsculas
	%subsection=TITLE,		% títulos de subseções convertidos em letras maiúsculas
	%subsubsection=TITLE,	% títulos de subsubseções convertidos em letras maiúsculas
	% -- opções do pacote babel --
	english,					% idioma adicional para hifenização
	%french,					% idioma adicional para hifenização
	%spanish,				% idioma adicional para hifenização
	brazil					% o último idioma é o principal do documento
	]{abntex2}

% ---------------------
% Pacotes OBRIGATÓRIOS
% ---------------------
\usepackage{lmodern}				% Usa a fonte Latin Modern			
\usepackage[T1]{fontenc}			% Selecao de codigos de fonte.
\usepackage[utf8]{inputenc}		% Codificacao do documento (conversão automática dos acentos)
\usepackage{lastpage}			% Usado pela Ficha catalográfica
\usepackage{indentfirst}			% Indenta o primeiro parágrafo de cada seção.
\usepackage{color}				% Controle das cores
\usepackage{graphicx,graphicx}	% Inclusão de gráficos
\usepackage{epsfig,subfig}		% Inclusão de figuras
\usepackage{microtype} 			% Melhorias de justificação
% ---------------------
		
% ---------------------
% Pacotes ADICIONAIS
% ---------------------
\usepackage{lipsum}						% Geração de dummy text
\usepackage{amsmath,amssymb,mathrsfs}	% Comandos matemáticos avançados 
\usepackage{setspace}  					% Para permitir espaçamento simples, 1 1/2 e duplo
\usepackage{verbatim}					% Para poder usar o ambiente "comment"
\usepackage{tabularx} 					% Para poder ter tabelas com colunas de largura auto-ajustável
\usepackage{afterpage} 					% Para executar um comando depois do fim da página corrente
\usepackage{url} 						% Para formatar URLs (endereços da Web)
\usepackage{amsthm}                                        % Para provas de teoremas
\usepackage{algorithm}
\usepackage{algpseudocode}
\usepackage{float}                                              %Para fixar floats

% --------------------------
% Codigos python

\usepackage{listings}
\usepackage{xcolor}
\definecolor{codegreen}{rgb}{0,0.6,0}
\definecolor{codegray}{rgb}{0.5,0.5,0.5}
\definecolor{codepurple}{rgb}{0.58,0,0.82}
\definecolor{backcolour}{rgb}{0.95,0.95,0.92}

\lstdefinestyle{mystyle}{
    backgroundcolor=\color{backcolour},   
    commentstyle=\color{codegreen},
    keywordstyle=\color{magenta},
    numberstyle=\tiny\color{codegray},
    stringstyle=\color{codepurple},
    basicstyle=\footnotesize,
    breakatwhitespace=false,         
    breaklines=true,                 
    captionpos=b,                    
    keepspaces=true,                 
    numbers=left,                    
    numbersep=5pt,                  
    showspaces=false,                
    showstringspaces=false,
    showtabs=false,                  
    tabsize=2
}

\lstset{style=mystyle}
% ---------------------

% ---------------------Î
% Pacotes de CITAÇÕES
% ---------------------
\usepackage[brazilian,hyperpageref]{backref}	% Paginas com as citações na bibl
% \usepackage[alf]{abntex2cite}				% Citações padrão ABNT (alfa)
%\usepackage[num]{abntex2cite}				% Citações padrão ABNT (numericas)
 \usepackage{cite}
% ---------------------

% Configurações de CITAÇÕES para abntex2
\include{extras/conf_citacoes}

% Inclusão de dados para CAPA e FOLHA DE ROSTO (título, autor, orientador, etc.)
% ---
% Informações de dados para CAPA e FOLHA DE ROSTO
% ---
\titulo{Compressão de Dados e Entropia no Contexto Linguístico}
\autor{Lucas Silva Amorim}
\local{Santo André - SP}
\data{Maio de 2022}
\orientador{Prof.ª Dr.ª Cristiane M. Sato}
\instituicao{%
  Universidade Federal do ABC -- UFABC
  \par
  Centro de Matemática, Computação e Cognição
  \par
  Bacharelado em Ciências da Computação.}
\tipotrabalho{Projeto de Graduação}
% O preambulo deve conter o tipo do trabalho, o objetivo,
% o nome da instituição e a área de concentração
\preambulo{\textbf{Projeto de Graduação} apresentado ao Centro de Matemática, Computação e Cognição, como parte dos requisitos necessários para a obtenção do Título de Bacharelado em Ciências da Computação.}
% ---

% Inclui Configurações de aparência do PDF Final
\include{extras/conf_pdf}

% O tamanho da identação do parágrafo é dado por:
\setlength{\parindent}{1.3cm}

% Controle do espaçamento entre um parágrafo e outro:
\setlength{\parskip}{0.2cm}  % tente também \onelineskip

% ---------------------
% Compila o indice
% ---------------------
\makeindex
% ---------------------

% --- 
% Definindo ambiente de criação de teoremas
% ---
\newtheorem{theorem}{Teorema}
\newtheorem{corollary}{Corolário}[theorem]
\newtheorem{lemma}[theorem]{Lema}

%%%%%%%%%%%%%%%%%%%%%%%%%%%
%%  INICIO DO DOCUMENTO  %%
%%%%%%%%%%%%%%%%%%%%%%%%%%%
\begin{document}

% Retira espaço extra obsoleto entre as frases.
\frenchspacing

% ----------------------------------------------------------
% ELEMENTOS PRÉ-TEXTUAIS (Capa, Resumo, Abstract, etc.)
% ----------------------------------------------------------
\pretextual

% Capa
\include{pretextual/capa}

% Folha de rosto (o * indica que haverá a ficha bibliográfica)
\imprimirfolhaderosto*

% Imprimir Ficha Catalografica
% \include{pretextual/catalografica}

% Inserir Folha de Aprovação
% \include{pretextual/assinaturas}

% Dedicatória
% % ---
% Dedicatória
% ---
\begin{dedicatoria}
   \vspace*{\fill}
   \centering
   \noindent
   \textit{ Aos verme que roeu as frias carnes de meu cadáver.} \vspace*{\fill}
\end{dedicatoria}
% ---

% Agradecimentos
% % ---
% Agradecimentos
% ---
\begin{agradecimentos}



Agradeço a Xuxa, meus pais, cachorro, gato e papagaio, por ...

Agradeço ao meu orientador, XXXXXXXXX, por todos os conselhos, pela paciência e ajuda nesse período.

Aos meus amigos ...

Aos professores ...

À XXXXXX pelo apoio financeiro para realização deste trabalho de pesquisa.

\end{agradecimentos}
%% ---

% Epígrafe
% \include{pretextual/epigrafe}

% Resumo e Abstract
% ---
% RESUMOS
% ---

% RESUMO em português
\setlength{\absparsep}{18pt} % ajusta o espaçamento dos parágrafos do resumo
\begin{resumo}
Em um cenário onde a informação é um item cada muito valioso, a quantidade de bytes de texto consumidos cresce diariamente.
Portanto, é muito importante diminuir o tamanho dos dados armazenados, tarefa que é feita pelos algoritmos de compressão.
Neste trabalho, exploramos e testamos algumas técnicas para potencializar a compressão de dados textuais.
A clusterização aliada à compressão baseada em \emph{palavras} se mostrou eficiente na melhoria da taxa de compressão dos algoritmos, conforme mostram os experimentos realizados.

 \textbf{Palavras-chaves}: compressão. texto. clusterização.
\end{resumo}

% ABSTRACT in english
\begin{resumo}[Abstract]
 \begin{otherlanguage*}{english}
In a scenario where information is a very valuable item, the amount of text bytes consumed grows daily.
Therefore, it is very important to reduce the size of the stored data, a task that is done by the compression algorithms.
In this work, we explore and test some techniques to enhance the compression of textual data.
The clustering combined with the compression based on \emph{words} proved to be efficient in improving the compression rate of the algorithms, as shown by the experiments carried out.
  
  

   \noindent 
   \textbf{Keywords}: compression. text. clustering.
 \end{otherlanguage*}
\end{resumo}

% Lista de ilustrações
% \pdfbookmark[0]{\listfigurename}{lof}
% \listoffigures*
% \cleardoublepage

% % Lista de tabelas
% \pdfbookmark[0]{\listtablename}{lot}
% \listoftables*
% \cleardoublepage

% % Lista de abreviaturas e siglas
% \begin{siglas}
%   \item[ABNT] Associação Brasileira de Normas Técnicas
%   \item[abnTeX] Normas para TeX
% \end{siglas}

% % Lista de símbolos
% \begin{simbolos}
%   \item[$ \Gamma $] Letra grega Gama
%   \item[$ \Lambda $] Lambda
%   \item[$ \zeta $] Letra grega minúscula zeta
%   \item[$ \in $] Pertence
% \end{simbolos}

% Inserir o SUMÁRIO
\pdfbookmark[0]{\contentsname}{toc}
\tableofcontents*
\cleardoublepage

% ----------------------------------------------------------
% ELEMENTOS TEXTUAIS (Capítulos)
% ----------------------------------------------------------
\textual
% Elementos textuais com numeração arábica
\pagenumbering{arabic}
% Reinicia a contagem do número de páginas
\setcounter{page}{1}

% Inclui cada capitulo da Dissertação
% ----------------------------------------------------------
% Introdução 
% Capítulo sem numeração, mas presente no Sumário
% ----------------------------------------------------------
\chapter*[Introdução]{Introdução}

Em um mundo cada vez mais informatizado, a demanda por armazenamento e transmissão de grandes quantidades de dados cresce exponencialmente.  
Algoritmos de compressão de dados são amplamente utilizados por protocolos de transmissão de informações \cite{MDN} e em sistemas de banco de dados \cite{MicDocs}, isso porque o processo de compressão permite a redução total de dados para representar uma certa informação.

Um algoritmo de compressão pode ser classificado como sendo \textbf{com perda}, quando reconstrói apenas uma aproximação do conteúdo original. 
Esses algoritmos são comumente utilizados na compressão de mídias como vídeos, imagens e áudio, onde a perda de alguns bits não tem uma interferência relevante na recuperação do conteúdo original. 
Em contrapartida, os algoritmos do tipo \textbf{sem perda}, são aqueles em que o conteúdo original que foi comprimido é totalmente recuperado após a descompressão, sendo utilizados principalmente para em textos, onde uma simples troca de caracteres pode comprometer o significado do conteúdo original. 

Existem diversos tipos de algoritmos de compressão sem perda, neste trabalho utilizaremos os \textbf{baseados em probabilidades} e \textbf{baseados em dicionários}. 
Os algoritmos \textbf{baseados em probabilidade} constroem o código a partir das probabilidades de ocorrência de cada símbolo dentro do texto. 
Já os \textbf{baseados em dicionários} tem como ideia central encontrar padrões nos dados substituindo esses padrões por tokens que ocupam "menos memória" \cite{Camb}.

Classificamos a compressão de dados como uma sub-área da Teoria da Informação, já que todos os resultados e limites teóricos para os algoritmos são fundamentados pela mesma. 
De fato, todo algoritmo de compressão assume que a mensagem possui um certo nível de redundância, e toma vantagem desse fato para representar a informação de maneira mais eficiente. 
Portanto, podemos supor que a \textbf{taxa de compressão} de um algoritmo está intimamente relacionada com a sua entropia (esse assunto será melhor detalhado durante em capítulos posteriores), o que nos dá um indício de que \textbf{minimizar a entropia}, pode \textbf{maximizar} a eficiência na compressão.

O foco deste trabalho, estará justamente em explorar de maneira empírica essa relação entre a entropia associada a um texto e a taxa de compressão, utilizando técnicas de pré-processamento para minimização da entropia. 
Vamos utilizar técnicas de clusterização de dados \cite{Goog} para minimizar a entropia da base que será experimentada a fim de otimizar o processo de compressão.
Através deste trabalho espera-se validar a hipótese de que a clusterização pode ser utilizada como um método de melhoria em algoritmos de compressão de textos, bem como apresentar os fundamentos teóricos que sustentam esta hipótese.

O texto está dividido em três partes.
 A primeira parte apresenta a definição de código, entropia e a relação entre esses conceitos, através de lemas e teoremas que fundamentam estes resultados.
 A segunda parte apresenta os algoritmos de compressão utilizados no trabalho (Huffman, Huffword, LZ77, WLZ 77), bem como alguns detalhes de implementação e otimizações que serão utilizadas posteriormente.
 Ainda nesta mesma parte, são apresentados alguns conceitos básicos e definições fundamentais para o entendimento do método de clusterização (\emph{k-means}) que será aplicado como pré-processamento.
 A parte final do trabalho apresenta o experimento que testa a clusterização como um método de melhoria na taxa de compressão dos algoritmos apresentados, detalhando as tecnologias e métodos utilizados, bem como os resultados obtidos.
 
 Observa-se que, algumas das demonstrações omitidas nos textos utilizados como base, foram incluídas por elaboração do autor deste texto (e devidamente indicadas como autorais), como parte do esforço intelectual para a construção deste trabalho.
 Por fim, a ideia geral do trabalho foi concebida para explorar os aspectos diversos do conhecimento adquirido durante o curso de Bacharelado em Ciência da Computação, esperamos que este texto possa transmitir isto ao leitor de maneira clara e objetiva.


% PARTE - Define a divisão do documento em partes (Não é obrigatório)
\part{Fundamentação Teórica}
% ---
\chapter{Conceitos, definições e resultados fundamentais em compressão de dados}
% ---
Este capítulo apresenta algumas definições, conceitos e resultados fundamentais para o entendimento das técnicas de compressão que serão discutidas em capítulos posteriores.

% -- Definições básicas de código
\section{Código}

Dado um conjunto $A$, usaremos a notação $A^+$ para definir o conjunto
que contém todas as cadeias não-vazias formadas pelas possíveis
combinações de $A$. Ou seja,
\begin{equation*}
  \emph{A}^+ = \Big\{
  (a_1,a_2,\dotsc, a_n) \colon n\in\mathbb{N}, \ n>0,
  \ a_i \in A \ \forall i \in \{1,\dotsc,n\} 
  \Big\}
\end{equation*}
Por simplicidade de notação, utilizaremos $a_1a_2\dotsm a_n$ para
denotar a sequência $(a_1,a_2,\dotsc, a_n)$ quando não houver
ambiguidade.

Um \textbf{alfabeto} $A$ é um conjunto finito. Chamaremos os elementos
de $A$ de \textbf{símbolos} ou \textbf{letras} e os elementos de $A^+$
de \textbf{cadeias} ou \textbf{palavras}.

Um \textbf{código} $C$ mapeia cada símbolo $m \in M$ para uma cadeia
em $W^+$. Mais precisamente, um código é uma função injetora de um
conjunto $M$ para $W^+$. O conjunto $M$ é chamado de \textbf{alfabeto
  de origem} e $W$ é chamado de \textbf{alfabeto código}. Chamaremos
cada cadeia na imagem de $C$ de \textbf{palavra-código}.

Dentro deste contexto, definimos o \textbf{comprimento} de uma palavra
$w\in W^+$, denotado por $l(w)$, como o inteiro positivo que
representa o tamanho da sequência $w$.

Tome como exemplo o alfabeto de origem $M = \{a, b, c\}$ e o alfabeto
código $W = \{0, 1\}$ composto somente por valores binários,
poderíamos definir um código $C$ da seguinte forma:

\begin{table}[!h]
   \centering
   \caption{Tabela do código C} \label{tab:vcode}
   \begin{tabular}{|l|c|c|c|c|c|c|r|}
        \hline
        \small{alfabeto de origem} & \small{palavra-código} \\ \hline
              a &   1   \\ \hline
              b &   01  \\ \hline
              c &   00  \\ \hline
  \end{tabular}
\end{table}
Neste exemplo, o \textbf{comprimento} da palavra-código associado à
letra ``b''~é dado por $l(C(b)) = l(01) = 2$.

As palavras-código associadas a cada símbolo podem ter um tamanho \emph{fixo} ou \emph{variável}.
Códigos nos quais palavras-código possuem um comprimento fixo são chamados de \textbf{códigos de comprimento fixo}, enquanto os que possuem alfabetos de comprimento variáveis são chamados \textbf{códigos de comprimento variável}. Note que o exemplo anterior é um código de comprimento variável. 
Provavelmente o exemplo mais conhecido de código de \textbf{comprimento fixo} seja código ASCII, que mapeia 64 símbolos alfa-numéricos (ou 256 em sua versão estendida) para palavras-código de 8 bits. 
Todavia, no contexto de compressão de dados procuramos construir códigos que podem variar em seu comprimento baseados na sua probabilidade associada, a fim de reduzir o tamanho médio da \emph{string} original ao codificá-la.

\subsection{Códigos unicamente decodificáveis e livres de prefixo}

%Um código é \textbf{distinto} se pode ser representado como uma função \textbf{bijetora}, i.e, $\forall$ $m_1$, $m_2$ $\in$ M, \emph{C($m_1$)} $\neq$ \emph{C($m_2$)}.
Dado um alfabeto de origem $M$, chamamos as palavras em $M^+$ de
\textbf{mensagens}. De fato, queremos codificar e decodificar
mensanges utilizando códigos e não somente símbolos isolados. Podemos
estender código para mensagens naturalmente da forma a seguir. Dado um
código $C\colon M\to W^+$, defina \textbf{código estendido} $C^+: M^+\to W^+$ por
\begin{equation*}
  C^+(m_1m_2\dotsc m_k) =
  \textrm{concat}(C(m1),C(m_2),\dotsc,C(m_k)), \text{ para todo }m_1m_2\dotsc m_k\in M^+,
\end{equation*}
onde $\textrm{concat}(\cdot)$ é a função de concatenação de sequências.


Um código $C\colon M\to W^+$ é dito \textbf{unicamente decodificável}
se $C^+$ é uma função injetora. Ou seja, toda mensagem é mapeada para
uma palavra única.

Um \textbf{código livre de prefixo} é um código em que nenhuma
palavra-código é prefixo de outra. Uma palavra $w_1\dotsm w_k$ é
\textbf{prefixo} de uma palavra $v_1\dotsm v_\ell$ se $k\leq \ell$ e
$w_1\dotsm w_k = v_1\dotsm v_k$. Ou seja, um código $C\colon M\to W^+$
é livre de prefixo se vale que, para quaisquer $m,m' \in M$ distintos,
$C(m)$ não é prefixo de $C(m')$.

Por exemplo, o código que possui sua imagem no conjunto de
palavras-código \emph{$W^+$} := $\{1, 01, 000, 001\}$ não possui
nenhuma cadeia que é prefixo de outra, portanto é considerado um
\textbf{código livre de prefixo}.  Códigos livres de prefixo não
apenas são unicamente decodificáveis como podem ser
\emph{decodificados instantaneamente}, pois, ao processar uma cadeia
de sequência de palavras-código podemos decodificar cada uma delas sem
precisar verificar o início da seguinte.

Todo código $C\colon M\to W^+$ pode ser representado por uma árvore
$T(C)$ onde
\begin{itemize}
\item as arestas são rotuladas por símbolos de $W$ e,
\item cada símbolo $m \in M$ é associado a um nó $v(m)$ da árvore,
\item para cada símbolo $m\in M$, a palavra-código $C(m)$ é obtida
  pelo caminho da raiz até $v(m)$ através da concatenação dos rótulos
  das arestas do caminho
\item cada folha da árvore é $v(m)$ para algum $m\in M$.
\end{itemize}
Note que, se $C$ é um código livre de prefixo, todos os símbolos em
$M$ são mapeados para folhas, pois se $v(m)$ está no caminho da folha
até $v(m')$ ($v(m)$ é ancestral de $m'$), então $C(m)$ é prefixo de
$C(m')$. A árvore a seguir representa o código da
Tabela~\ref{tab:vcode} (claramente um código livre de prefixo).

\begin{figure}[h]
   \centering
   \includegraphics[scale=0.75]{figs/prefixtree.png}
    \caption{Árvore de um código livre de prefixo}
    \label{fig:prefixt}
 \end{figure}
% ----------------------------------------------------------------------------
% Probabilidade -------------------------
\section{Relações fundamentais com a Teoria da Informação}
A codificação é comumente divida em duas componentes diferentes: \emph{modelo} e \emph{codificador}. 
O \emph{modelo} identifica a distribuição de probabilidade das mensagens baseado em sua semântica e estrutura. 
O \emph{codificador} toma vantagem de um possível \emph{bias} apontado pela modelagem, e utiliza as probabilidades associadas para reduzir a quantidade de dados necessária para representar a mesma informação (substituindo as mensagens que ocorrem com maior frequência por símbolos menores).
Isso significa que, a codificação está diretamente ligada as probabilidades associadas a cada mensagem.

Nesta seção, vamos construir o embasamento teórico necessário para entender a relação entre as probabilidades associadas e o comprimento das mensagens, e consequentemente criar uma noção dos parâmetros que devem ser maximizados para alcançar uma codificação eficiente.

\subsection{Distribuição de Probabilidade e Esperança}
Dado um experimento e um espaço amostral $\Omega$, uma \textbf{variável aleatória} \emph{X} associa um número real a cada um dos possíveis resultados em $\Omega$. Em outras palavras, \emph{X} é uma função que mapeia os elementos do espaço amostral para números reais. Uma variável aleatória é chamada \textbf{discreta} quando os valores dos experimentos associados a ela são números finitos ou ao menos infinitos que podem ser contados.
Podemos descrever melhor uma variável aleatória, atribuindo probabilidades sobre os valores que esta pode assumir. Esses valores são atribuídos pela \textbf{função de densidade de probabilidade}, denotada por \emph{$p_X$}. Portanto, a probabilidade do evento \{ \emph{X} = \emph{x} \} é a função de distribuição de probabilidade aplicada a x, \emph{i.e}, \emph{$p_X(x)$}.
\begin{equation} \label{eq:dist_prob_def}
p_X(x) = P(\{X = x\})
\end{equation}

Note que, a variável aleatória pode assumir qualquer um dos valores no espaço amostral que possuem uma probabilidade $\emph{P} > 0$, portanto
\begin{equation} \label{eq:dist_prob_sum}
\sum_{x ~\in ~im_X}^{}p_X(x) = 1.
\end{equation}

O \textbf{valor esperado} (ou \textbf{esperança}) da variável aleatória \emph{X} é definido como
\begin{equation} \label{eq:exp_val}
\textbf{E}[X] = \sum_{x ~\in ~im_X}^{} xp_X(x).
\end{equation}
%--------------------------------------
% ------------------------------ Entropia
\subsection{Comprimento médio do código}
Seja \emph{p} a distribuição de probabilidade associada ao alfabeto de origem \emph{M}. Assuma que \emph{C} é um código tal que \emph{C(m)} = \emph{w}, definimos o \textbf{tamanho médio} de \emph{C} como:
\begin{equation} \label{eq:code_len}
l_a (C) = \sum_{m \in M}^{} p(m) l(C(m))
\end{equation}

Um código \emph{C} unicamente decodificável é \textbf{ótimo} se $l_a(C)$ é mínimo, isto é, para qualquer código unicamente decodificável \emph{C'} temos que:
\begin{equation} \label{eq:code_len_optimal}
l_a(C) \leq l_a(C')
\end{equation}

\subsection{Entropia}
A \textbf{entropia de Shannon} aplica as noções de entropia física (que representa a aleatoriedade de um sistema) à Teoria da Informação. Dado um espaço de probabilidade \emph{X} e a função \emph{p} sendo a distribuição de probabilidade associada a \emph{X}, definimos \textbf{entropia} como:
\begin{equation} \label{eq:entropy}
H(X, p) = \sum_{x \in X}^{} p(x) \log_2 \frac{1}{p(x)}
\end{equation}

Por esta definição temos que quanto menor o \emph{bias} da função de distribuição de probabilidade relacionada ao sistema, maior a sua entropia. Em outras palavras, a entropia de um sistema esta intimamente ligada a sua  "desordem". 

Shannon (incluir referencia papper do Shannon) aplica o mesmo conceito de entropia no contexto da Teoria da Informação, "substituindo" o conjunto de estados \emph{S} pelo conjunto de mensagem \emph{M}, isto é, \emph{M} é interpretado como um conjunto de possíveis mensagens, tendo como \emph{p(m)} a probabilidade de $m \in M$.
Baseado na mesma premissa, Shannon mede a informação contida em uma mensagem da seguinte forma:
\begin{equation} \label{label:info_quantity}
\emph{i(m)} = \log_2 \frac{1}{p(m)}.
\end{equation}

\subsection{Comprimento de Código e Entropia}
Nas seções anteriores, o comprimento médio de um código  foi definido em função da distribuição de probabilidade associada ao seu alfabeto de origem.
Da mesma forma, as noções de \textbf{entropia} relacionada a um conjunto de mensagens, têm ligação direta com as probabilidades associadas a estas. 
A seguir, será mostrado como podemos relacionar o comprimento médio de um código a sua entropia através da \textbf{Desigualdade de Kraft-McMillan} , e por consequência estabelecer uma relação direta entre \textbf{a entropia de um conjunto de mensagens e a otimalidade do código associada a estas mensagens}.

\begin{lemma}[Desigualdade de Kraft-McMillan] 
\textbf{Kraft.}. Para qualquer conjunto \emph{L} de comprimento códigos que satisfaça:
\begin{align*}
\sum_{l \in L}^{} 2^{-l} \leq 1.
\end{align*}
Existe ao menos um código livre de prefixo tal que, $|w_i|$ = $l_i$ ~,~$\forall w \in W^+$.

\textbf{Kraft-McMillan.} Para todo código binário unicamente decodificável \textbf{\emph{C} : $\emph{M} \rightarrow \emph{W}^+$} .
\begin{equation*}
\sum_{w \in W^+}^{}2^{-l(w)} \leq 1.
\end{equation*}


\begin{proof}

\item \textbf{Desigualdade de Kraft}
Sem perda de generalidade, suponha que os elementos de \emph{L} estão ordenados de maneira em que:
\begin{align*}
l_1 \leq l_2 \leq ... \leq l_n
\end{align*}
Agora vamos construir um código livre de prefixo em uma ordem crescente de tamanho, de maneira em que $l(w_i) = l_i$. Sabemos que um código é livre de prefixo se e somente se ,existe uma palavra-código $w_j$  tal que nenhuma das palavras-código anteriores $(w_1, w_2, w_{j-1})$ são prefixo de $w_j$.

Sem as restrições de prefixo, uma palavra-código de tamanho $l_j$ poderia ser construída de $2^{l_j}$ maneiras diferentes. Com a restrição apresentada anteriormente, considerando uma palavra $w_k$ anterior a $w_j$ (i.e, $k < j$), existem $2^{l_j - l_k}$ possíveis palavras-código em que $w_k$ seria um prefixo, e que portanto não podem pertencer ao código. Chamaremos tal conjunto de "palavras-código proibidas". Vale notar que os elementos do conjunto de palavras-código proibidas são excludentes entre si, pois se duas palavras-código menores que \emph{j} fossem prefixo da mesma palavra-código, elas seriam prefixos entre si. 

Dito isto, podemos definir o tamanho do conjunto de palavras-código proibida para $w_j$.
\begin{equation*}
\sum_{i=1}^{j-1} 2^{l_j - l_i}
\end{equation*}

A construção do código livre de prefixo é possível se e somente se, existir ao menos uma palavra-código de tamanho $j > 1$ que não está contida no conjunto das palavras-código proibidas.
\begin{equation*}
2^{l_j} > \sum_{i=1}^{j-1} 2^{l_j - l_i}
\end{equation*}

Como o domínio do problema apresentado está restrito aos inteiros não negativos, podemos afirmar que:
\begin{align*}
2^{l_j} > \sum_{i=1}^{j-1} 2^{l_j - l_i} &= 2^{l_j} \geq \sum_{i=1}^{j-1} 2^{l_j - l_i} + 1 \\
&= 2^{l_j} \geq \sum_{i=1}^{j} 2^{l_j - l_i} \\
&= 1 \geq \sum_{i=1}^{j} 2^{-l_i} \\
&=  \sum_{i=1}^{j} 2^{-l_i} \leq 1
\end{align*}

Substituindo \emph{n} em \emph{j}, chegamos a desigualdade de Kraft.
\begin{equation*}
\sum_{l \in L}^{} 2^{-l} \leq 1.
\end{equation*}

Note que os argumentos utilizados para a construção da prova possuem dupla-equivalência, portando concluem a prova nos dois sentidos.

\item \textbf{Kraft-McMillan}
Suponha um código \textbf{unicamente decodificável} \emph{C} qualquer, e faça $l_{max}$ = $\max_{w}l(w)$.

Agora considere uma sequência de \emph{k} palavras-código de \textbf{\emph{C} : $\emph{M} \rightarrow \emph{W}^+$} (onde \emph{k} é um inteiro positivo). Observe que:

\begin{align*}
(\sum_{w \in W^+}^{}2^{-l(w)})^k &= (\sum_{w_1}^{}2^{-l(w_1)}) \cdot (\sum_{w_2}^{}2^{-l(w_2)}) \cdot ... \cdot (\sum_{w_k}^{}2^{-l(w_k)}) \\
&= \sum_{w_1}^{} \sum_{w_2}^{} ... \sum_{w_k}^{} \prod_{j = 1}^{k} 2^{-l(w_j)} \\
&= \sum_{w_1,..., w_k}^{} 2^{- \sum_{j=1}^{k} l(w_j)} \\
&= \sum_{w_k}^{} 2^{-l(w^k)} \\
&= \sum_{j=1}^{k \cdot l_{max}} |\{w_k | l(w_k) = j\}| \cdot 2^{-j}.
\end{align*}

Sabemos que existem $2^{j}$ palavras-código de tamanho \emph{j}, isto é, $|\{w_k | l(w_k) = j\}| = 2^j$. Para que o código seja unicamente decodificável obtemos o seguinte limite superior:
\begin{equation*}
(\sum_{w \in W^+}^{}2^{-l(w)})^k \leq \sum_{j=1}^{k \cdot l_{max}} 2^j \cdot 2^{-j} = k \cdot l_{max}
\end{equation*}

Logo,
\begin{equation*}
\sum_{w \in W^+}^{}2^{-l(w)} \leq (k \cdot l_{max})^ \frac{1}{k}
\end{equation*}

Note que a desigualdade é valida para qualquer $k > 0$ inteiro. Aproximando \emph{k} ao infinito, obtemos a desigualdade de Kraft-McMillan.

\begin{equation*}
\sum_{w \in W^+}^{}2^{-l(w)} \leq \lim_{k\to\infty} (k \cdot l_{max})^ \frac{1}{k} = 1.
\end{equation*}

\end{proof}
\end{lemma}

\begin{lemma}[Entropia como limite inferior para o comprimento médio] Dado um conjunto de mensagens \emph{M} associado a uma distribuição de probabilidades \emph{p} e um código unicamente decodificável \emph{C}.
\begin{equation*}
H(M, p) \leq l_a(C)
\end{equation*}

\begin{proof}
Queremos provar que $H(M, p) - l_a(C) \leq 0$, dado que  $H(M, p) \leq l_a(C) \Leftrightarrow H(M, p) - l_a(C) \leq 0$.

Substituindo a equação~\ref{eq:entropy} em \emph{H(M, p)} e~\ref{eq:code_len} em \emph{$l_a(C)$}, temos:

\begin{align*}
H(M, p) - l_a(C) &= \sum_{m \in M}^{}p(s) \log_2 \frac{1}{p(m)}  - \sum_{m \in M, w \in W^+}^{}p(m) l(w) \\
&= \sum_{m \in M, w \in W^+}^{}p(m) \left(  \log_2 \frac{1}{p(m)} - l(w) \right) \\
&= \sum_{m \in M, w \in W^+}^{}p(m) \left(  \log_2 \frac{1}{p(m)} - \log_2 2^{l(w)} \right) \\
&= \sum_{m \in M, w \in W^+}^{}p(m) \log_2 \frac{2^{-l(w)}}{p(m)}
\end{align*}

A \textbf{Desigualdade de Jansen} afirma que se uma função \emph{f(x)} é côncava, então $\sum_{i}{}p_i~f(x_i) \leq f(\sum_{i}{}p_i~x_i)$. Como a função $\log_2$ é côncava, podemos aplicar a Desigualdade de Jansen ao resultado obtido anteriormente.

\begin{equation*}
\sum_{m \in M, w \in W^+}^{}p(m) \log_2 \frac{2^{-l(w)}}{p(m)}  \leq \log_2(\sum_{m \in M, w \in W^+}{}2^{-l(w)})
\end{equation*}

Agora aplicamos a desigualdade de Kraft-McMillan, e concluímos que:

\begin{equation*}
H(M, p) - l_a(C) \leq \log_2(\sum_{m \in M, w \in W^+}{}2^{-l(w)}) \Rightarrow H(M, p) - l_a(C) \leq 0.
\end{equation*}


\end{proof}

\end{lemma}

\begin{lemma}[Entropia como limite superior para o comprimento médio de um código livre de prefixo ótimo] Dado um conjunto de mensagens \emph{M} associado a uma distribuição de probabilidades \emph{p} e um código livre de prefixo ótimo \emph{C}.
\begin{equation*}
l_a(C) \leq H(M, p) + 1
\end{equation*}

\begin{proof}
Sem perda de generalidade, para cada mensagem $m \in M$ faça \emph{l(m)} = $\left \lceil{\log_2 \frac{1}{p(m)} }\right \rceil $. Temos que:
\begin{align*}
\sum_{m \in M}^{} 2^{-l(m)} &= \sum_{m \in M}^{} 2^{-\left \lceil{\log_2 \frac{1}{p(m)} }\right \rceil} \\
&\leq \sum_{m \in M}^{} 2^{-{\log_2 \frac{1}{p(m)} }} \\
&= \sum_{m \in M}^{} p(m) \\
&= 1
\end{align*}

De acordo com a desigualdade de Kraft-McMillan existe um código livre de prefixo \emph{C'}  com palavras-código de tamanho \emph{l(m)}, portanto:
\begin{align*}
l_a(C') &=  \sum_{m \in M', w \in W'^+}^{}p(m) l(w) \\
&=  \sum_{m \in M', w \in W'^+}^{}p(m) \left \lceil{\log_2 \frac{1}{p(m)} }\right \rceil \\
&\leq \sum_{m \in M', w \in W'^+}^{}p(m) (1 + \log_2 \frac{1}{p(m)}) \\
&= 1 +  \sum_{m \in M', w \in W'^+}^{}p(m) \log_2 \frac{1}{p(m)} \\
&= 1 + H(M)
\end{align*}

Pela definição de código livre de prefixo ótimo, $l_a(C) \leq l_a(C')$, isto é:
\begin{equation*}
l_a(C) \leq H(M, p) + 1 
\end{equation*}
\end{proof}
\end{lemma}


% ------------------- End of chapter 1 -----------------------------

% --- Guardando para exemplo
% A formatação das referências bibliográficas conforme as regras da ABNT são um
% dos principais objetivos do \abnTeX. Consulte os manuais
% \citeonline{abntex2cite} e \citeonline{abntex2cite-alf} para obter informações
% sobre como utilizar as referências bibliográficas.

% %-
% \subsection{Acentuação de referências bibliográficas}
% %-

% Normalmente não há problemas em usar caracteres acentuados em arquivos
% bibliográficos (\texttt{*.bib}). Na~\autoref{tabela-acentos} você encontra alguns exemplos das conversões mais importantes. Preste atenção especial para `ç' e `í'
% que devem estar envoltos em chaves. A regra geral é sempre usar a acentuação
% neste modo quando houver conversão para letras maiúsculas.

% \begin{table}[htbp]
% \caption{Tabela de conversão de acentuação.}
% \label{tabela-acentos}

% \begin{center}
% \begin{tabular}{ll}\hline\hline
% acento & \textsf{bibtex}\\
% à á ã & \verb+\`a+ \verb+\'a+ \verb+\~a+\\
% í & \verb+{\'\i}+\\
% ç & \verb+{\c c}+\\
% \hline\hline
% \end{tabular}
% \end{center}
% \end{table}


% ---
% \section{Deu pau em algo?}
% ---

% Consulte a FAQ com perguntas frequentes e comuns no portal do \abnTeX:
% \url{https://code.google.com/p/abntex2/wiki/FAQ}.

% Inscreva-se no grupo de usuários \LaTeX:
% \url{http://groups.google.com/group/latex-br}, tire suas dúvidas e ajude a galera se tiver tudo certo.





\part{Algoritmos de compressão e pré-processamento}
% ---
\chapter{Algoritmos de compressão sem perda}
% ---

Com no que foi apresentado no capítulo anterior, a compressão de dados pode ser vista como uma sub-área da teoria da informação. O principal objeto de estudo é o desenvolvimento métodos e ferramental (ou otimização dos já existentes) que reduzam a quantidade de dados necessários para representar uma determinada informação.

Os algoritmos de compressão podem ser categorizadas em duas diferentes classes: os de compressão \textbf{com perda} e \textbf{sem perda}. Os \textbf{algoritmos de compressão sem perda} admitem uma baixa porcentagem de perda de informações durante a codificação para obter maior performance, muito uteis na transmissão de dados em streaming por exemplo. Nos \textbf{algoritmos de compressão com perda} o processo de codificação deve ser capaz de recuperar os dados em sua totalidade, geralmente utilizados em casos onde não pode haver perda de informações (como por exemplo, compressão de arquivos de texto).

\pagebreak

\section{Código de Huffman}
O \textbf{algoritmo de Huffman} (desenvolvido por David Huffman em 1952) é um dos componentes mais utilizados em algoritmos de compressão sem perda, servindo como base para algoritmos como o Deflate (utilizado amplamente na web).
Os códigos gerados a partir do algoritmos de Huffman são chamados \textbf{Códigos de Huffman}.

O código de Huffman é descrito em termos de como ele gera uma árvore de código livre de prefixo. Considere o conjunto de mensagens \emph{M}, com $p_i$ sendo a probabilidade associada a $m_i$

\begin{algorithm}[H]
\caption{Algoritmo de Huffman} \label{alg:huff}
\begin{algorithmic}
	\State $Forest \gets \emph{[]}$\\
	\ForAll{$m_i \in M$} \Comment{Inicializando floresta}
		\State $T \gets newTree()$
		\State $node \gets newNode()$
		\State $node.weight \gets p_i$ \Comment{$w_i = p_i$}
		\State $T.root \gets node$
		\State $Forest.append(T)$ \Comment{Adiciona um nova arvore mna floresta}
	\EndFor \\
	
	\While{$Forest.size > 1$}
		\State $T1 \gets ExtractMin(Forest)$ \Comment{Retorna a árvore cuja raiz é mínima, e retira da floresta}
		\State $T2 \gets ExtractMin(Forest)$
		\State $HTree \gets newTree()$
		\State $HTree.root \gets newNode()$ \\
		\State $HTree.root.left \gets T1.root$
		\State $HTree.root.right \gets T2.root$
		\State $HTree.root.weight \gets T1.root.weight + T2.root.weight$
		\State Forest.append(HTree) 
	\EndWhile
\end{algorithmic}
\end{algorithm}

\subsection{Análise Assintótica}
Seja \emph{n} o tamanho do conjunto de mensagens \emph{M}. Para que o algoritmo percorra toda a floresta, formada por uma árvore para cada $\emph{m} \in \emph{M}$, serão necessárias \emph{n} interações.
Considerando que as funções \emph{ExtractMin()} e \emph{.append()} foram construídas a partir de uma fila de prioridades de \textbf{heap}, o algoritmo será executado em  $O(n \log_2 n)$.

\subsection{Corretude}
O teorema a seguir (escrito por Huffman) mostra a principal propriedade do Algoritmo de Huffman, os códigos de Huffman são códigos ótimos e livres de prefixo.


\begin{lemma} \label{lemma:dist_prob_avg_size} Seja \emph{C} um código ótimo livre de prefixo, com $\{ p_1, p_2,..., p_n\}$ sendo a distribuição de probabilidades associada ao código. Se $p_i > p_j$ então $l(w_i) \leq l(w_j)$

\begin{proof} 
Assuma que $l(c_i) > l(c_j)$. Agora vamos construi um novo código \emph{C'}, trocando $w_i$ por $w_j$. Dado $l_a$ como o comprimento médio do código \emph{C}, o código \emph{C'} terá o seguinte comprimento:
\begin{align*}
l'_a &= l_a + p_j(l(w_i) - l(w_j)) + p_i(l(w_j) - l(w_i)) \\
&= l_a + (p_j - p_i)(l(w_i) - l(w_j)) 
\end{align*}

Pelas suposições feitas anteriormente o termo $(p_j - p_i)(l(w_i) - l(w_j))$ seria negativo, contradizendo o fato do código \emph{C} ser um código ótimo e livre de prefixo (pois neste caso $l'_a > l_a$).

\textbf{Nota*} : Perceba que em uma árvore de Huffman, o tamanho da palavra código $w_i$ também representa seu nível na árvore.
\end{proof}
\end{lemma}

\begin{theorem} O algoritmo de Huffman gera um código ótimo livre de prefixo.
\begin{proof}
A prova se dará por indução sobre o número de mensagens pertencentes ao código. Vamos mostrar que se o Algoritmo de Huffman gera um código livre de prefixo ótimo para qualquer distribuição de probabilidades com \emph{n} mensagens, então o mesmo ocorre para \emph{n + 1} mensagens.

\item \textbf{Caso Base.} Para n = 2 o teorema é trivialmente satisfeito considerando um código que atribui um bit pra cada mensagem do código.

\item \textbf{Passo indutivo}. Pelo lema \ref{lemma:dist_prob_avg_size} sabemos que as menores probabilidades estão nos menores níveis da árvore de Huffman (por ser uma árvore completa binário, o seu menor nível deve possuir ao menos dois nós). Já que esses nós possuiriam o mesmo tamanho, podemos muda-los de posição sem afetar o tamanho médio do código, concluindo assim que estes são nós \textbf{irmãos}.\\
Agora defina um conjunto de mensagens \emph{M} de tamanho \emph{n + 1} onde T é a árvore de prefixo ótima construída a partir do Algoritmo de Huffman aplicado em \emph{M}. Vamos chamar os dois nós de menor probabilidade na árvore de \emph{x} e \emph{y} (que pelo argumento anterior, são nós irmãos). Vamos construir uma nova árvore \emph{T'} a partir de \emph{T} removendo os nós \emph{x} e \emph{y}, fazendo assim que o pai destes nós, que chamaremos de \emph{z}, seja o de menor probabilidade (de acordo com a definição do Algoritmo de Huffman, $p_z = p_y + p_x$). Considere \emph{k} como a profundidade de \emph{z}, temos:

\begin{align*}
l_a(T) &= l_a(T') + p_x(k + 1) + py(k + 1) - p_z k \\
&= l_a(T') + p_x + p_y
\end{align*}

Sabemos pela hipótese de indução que $l_a(T')$ é mínimo, pois \emph{T'} tem o tamanho \emph{n} e foi gerada pelo algoritmo de Huffman. Note que independente da ordem que forem inseridos, os nós \emph{x} e \emph{y} irão adicionar a constante $p_z = p_x + p_y$ no peso médio do código. Como $l_a(T')$ é mínimo para um conjunto de mensagens de tamanho \emph{n} e seu nó de menor peso tem distribuição de probabilidade $p_z$, $l_a(T)$ também é mínimo para o conjunto de mensagens \emph{M} e logo \emph{T} é ótimo e livre de prefixo. 
\end{proof}
\end{theorem}



% PARTE
\part{Aplicação: Melhoria na compressão de dados textuais através da Clusterização}
\chapter{Clusterização como pré-processamento na compressão de textos}
Nos capítulos anteriores, construímos a estrutura teórica necessária para compreender a \emph{compressão de texto} e sua relação com a \emph{teoria da informação}.
Fica claro que os algoritmos apresentados, tomam vantagem de alguma redundância presente na mensagem para representar a informação de maneira mais eficiente.
Isto é, a \textbf{eficiência} destes algoritmos está intimamente ligada com a \textbf{entropia} do texto a ser comprimido,
 nos dando um indício de que \textbf{minimizar} a entropia, pode ser uma forma de \textbf{maximizar} a \textbf{taxa de compressão}.
 
O capítulo~\ref{cap:clus} apresenta uma técnica para organizar o texto em \emph{grupos de mensagens similares} (\textbf{clusterização}), o que significa diminuir a variação de termos dentro de cada cluster (ou reduzir a sua \textbf{entropia}).
Neste capítulo, exploraremos de maneira empírica a relação entre a entropia associada a um texto e a taxa de compressão dos algoritmos apresentados no capítulo~\ref{cap:comp}.

A seguir, será descrito um experimento que utiliza a \textbf{clusterização} como pré-processamento de uma base de dados textual, visando uma melhora na performance dos algoritmos de compressão \emph{sem perda}.
Foram realizados experimentos utilizando ou não o pré-processamento, afim de medir o impacto da clusterização para os algoritmos testados (que serão listados em seções posteriores).

Todo o projeto foi desenvolvido em \emph{Python} (principalmente ao seu vasto ferramental para manipulação de dados) e o código está disponível no GitHub [TODO REF GITHUB].
O código fonte está organizado nas seguintes camadas:
\begin{itemize}
	\item \textbf{dataset}: Contém os arquivos \emph{.csv} que serão carregados como base de dados.
	\item \textbf{compressors}: Classes e scripts com a implementação dos codificadores utilizados.
	\item \textbf{stackexchange\_compression\_experiments.ipynb}: \emph{Python notebook} com o código fonte de todo o experimento (pré-processamento dos dados, particionamento das bases, aplicação dos algoritmos de compressão, plot de gráficos, entre outros).
\end{itemize}
\pagebreak

\section{Escolha e tratamento de dados}
Para a realização do experimento, a escolha da base de dados é uma etapa primordial.
É necessária uma base de dados robusta, com uma grande quantidade de \textbf{texto} e que permita a criação de \emph{clusters} contendo assuntos diversos.

Para o experimento, foram selecionados os dados do desafio ``\emph{Transfer Learning o Stack Exchange Tags}''  disponível no \emph{kaggle}.
Trata-se de um conjunto de dados extraídos do website \emph{Stack Exchange}, basicamente um \textbf{fórum online} sobre os mais variados assuntos (ou seja, alta quantidade de textos de diferentes naturezas).

O \emph{dataset} tem como principais informações o título das questões, o conteúdo da questões (em formato HTML) e as \emph{tags} que classificam o conteúdo em diversos tópicos (biologia, culinária, criptografia, robótica e viagem).
O \emph{dataset} original possui um tamanho aproximado de \textbf{50MB}. 

\subsection{Criação do dataframe principal}
Os dados extraídos do \emph{kaggle} estão originalmente em arquivos \emph{csv}, um arquivo para cada tópico.
Portanto, para executar o experimento de compressão, precisamos unir os arquivos em um único \emph{dataframe}.
Para cada arquivo, selecionamos \textbf{1500} linhas (devido as limitações do hardware utilizado para o teste), o \emph{dataframe} final gerado possuí aproximadamente \textbf{7MB} de informação.


\begin{lstlisting}[language=Python, caption=Carregando base de dados]
# Loading dataframe
def load_dataset(dataset_name):
    f_path = 'dataset/stack-exchange-tags/{dataset}.csv'.format(dataset = dataset_name)
    full_path = os.path.join(os.getcwd(), f_path)
    return p.read_csv(full_path, index_col=0, nrows=1500)

ds_names = ['biology', 'cooking', 'crypto', 'diy', 'robotics', 'travel']
frames = [load_dataset(name) for name in ds_names]
df = p.concat(frames)
\end{lstlisting}

\subsection{Sanitização das colunas}
Como parte da preparação dos dados para o experimento, realizamos a sanitização das colunas do \emph{dataframe}.
Nessa etapa, removemos as \emph{stopwords}, pontuações, tags \emph{html} e outras informações que podem ser problemáticas para o processo de clusterização.
Vale salientar que, a sanitização é aplicada apenas como pré-processamento para a clusterização (basicamente, utilizaremos a informação para particionar os dados antes de comprimi-los).

\begin{lstlisting}[language=Python, caption=Sanitização do texto]
from wordcloud import STOPWORDS
import re\
stop_words = set(STOPWORDS)

# # Sanitizing columns
def sanitize_column(data):
    # convert to lower
    data = data.lower()
    # strip html
    data = re.sub(r'\<[^<>]*\>','',data.lower())
    #removing pontuation
    data = re.sub(r'[^a-zA-Z0-9]',' ',data)
    # Remove new lines and tabs
    data = re.sub(r'\s',' ',data)

    #strip data
    data = data.strip()

    #Spling words
    data = data.split()
    
    #Remove extra blank space
    data = list(filter(lambda s: s != ' ', data))

    # Remove stop words
    data =  list(filter(lambda s: s not in stop_words,data))

    #remove single chars words
    data = list(filter(lambda s: len(s) > 1, data))

    return data
def array_column_to_text(column):
    text = ''
    for w in column:
        for s in w:
            text += ' ' + s 
    return text
df['sanitezed_content'] = df['content'].apply(sanitize_column)
\end{lstlisting}

A biblioteca \textbf{worldcloud} nos permite visualizar a frequência de cada palavras de maneira gráfica.
A imagem a seguir mostra a representação dos conteúdo após a sanitização, dando destaque para as palavras mais frequentes.

 \begin{figure}[H]
   \centering
   \includegraphics[scale=0.50]{figs/wordcloud1.png}
    \caption{Cluster de palavras}
    \label{fig:wordcloud1}
 \end{figure}

\section{Compressão de textos}
Os experimentos foram realizados com dois algoritmos clássicos de compressão \emph{sem perda} e suas versões baseadas em palavras (\textbf{Huffman}, \textbf{Huffword}, \textbf{LZ77}, \textbf{WLZ77}).
Com estes algoritmos, teremos uma comparação do efeito da clusterização entre um algoritmo baseado em \textbf{probabilidades} (Huffman)  e outro baseado em \textbf{dicionário} (LZ77).
Espera-se que com a versão baseado em palavras, o impacto da clusterização na taxa de compressão seja ainda maior (já que os clusters são formados a nível de palavras).
A seguir, serão descritos alguns detalhes específicos da implementação dos compressores para o experimento (principalmente os detalhes não contidos no algoritmo formal apresentado no Capítulo~\ref{cap:comp}).

\subsection{Classes auxiliares}
Para facilitar a reutilização de código na aplicação dos diferentes compressores, criamos a classe auxiliar (\emph{TextCompressor}).
Nesta classe são expostos os método \emph{encode(text)} e \emph{decode()}, que servem como um ``contrato'' para a implementação dos algoritmos de compressão.

\begin{lstlisting}[language=Python, caption=Classe TextCompressor]
class TextCompressor:
    def __init__(self):
        self.originaltext = None
        self.stats = None
    def encode(self, text):
        self.originaltext = text
        pass
    def decode(self):
        pass
\end{lstlisting}

Cada \emph{classe} específica de compressor herda da classe base \emph{TextCompressor}, criando sua própria implementação dos métodos \emph{encode} e \emph{decode}.

Outra classe auxiliar implementada foi a \emph{CompressionStats}. Trata-se de uma interface para padronizar a leitura das métricas sobre os compressores.
\begin{lstlisting}[language=Python, caption=Implementaçào da classe base CompressionStats]
class CompressionStats:
    def _avg_code(self):
        return (self.compressedtextsize / self.originaltextsize) * 8
    def _compression_rate(self):
        return 100 - (self.compressedtextsize / self.originaltextsize) * 100
    def __init__(self, originaltext, compressedtext):
        self.originaltextsize = len(originaltext)
        self.compressedtextsize = len(compressedtext)
    def __str__(self) -> str:
        return stats_text.format(csize=self._avg_code(), osize=self.originaltextsize, nsize=self.compressedtextsize, crate=self._compression_rate())
\end{lstlisting}

\subsection{Compressão baseada em palavras}
Para a implementação dos algoritmos baseados em \textbf{palavras}, boa parte do código fonte utilizado nos algoritmos ``canônicos'' foram reutilizados.
Como o \emph{Python} é uma linguagem de \textbf{tipagem dinâmica}, a implementação ``canônica'' interpreta a entrada como um vetor de \emph{símbolos} (sem necessariamente distinguir o tipo de símbolo).
Isso permite uma maior flexibilidade para o reaproveitamento de grande parte das funções escritas, independente do alfabeto de origem utilizado. 

Conforme descrito no Capítulo~\ref{cap:comp}, o \emph{Huffword} computa as \emph{palavras} e \emph{separadores} como duas entradas distintas.
Para dividir a entrada entre \emph{palavras} e \emph{separadores}, utilizamos a biblioteca \emph{re}, que permite executar buscas em \emph{strings} utilizando \emph{regex}.
Basicamente, executamos uma busca na entrada pelas \emph{palavras} e atribuímos à variável \emph{words} (o mesmo vale para os separadores, que foram atribuídos a variável \emph{non\_words}).

\begin{lstlisting}[language=Python, caption=Função \emph{encode} para o \emph{Huffword}]
def _build_huffword_code(seq):
    freqs = frequency_dictionary(seq)
    huff_tree = canonical.build_huff_tree(freqs)
    code_table= canonical.build_code_table(huff_tree)
    return code_table
    
def huffword_encode(text):
    # Words huff tree
    words = re.findall(r'\w+', text)
    words_code = _build_huffword_code(words)

    # Non words huff tree
    nonwords = re.findall(r'\W+', text)
    nonwords_code = _build_huffword_code(nonwords)

    encoded_string = ""

    # When text start with non-word append 0, otherwise append 1
    starts_with = 0
    if text.startswith(words[0]):
        starts_with += 1
    
    # Append starts with as the first char
    encoded_string += str(starts_with)

    # Encode intercalating words and nonwords
    w_index = 0
    nw_index = 0

    #When starts with non words
    if not starts_with:
        encoded_string += nonwords_code[nonwords[0]]
    
    while w_index < len(words) or nw_index < len(nonwords):
        if w_index < len(words):
            word = words[w_index]
            encoded_string += words_code[word]
            w_index += 1
        
        if nw_index < len(nonwords):
            nonword = nonwords[nw_index]
            encoded_string += nonwords_code[nonword]
            nw_index += 1
    
    return {'encoded' : encoded_string,
            'words_meta' : (words, words_code),
            'non_words_meta': (nonwords, nonwords_code)}
\end{lstlisting}



\section{Partição de dados}
Para o propósito do experimento, a base de dados foi comprimida a partir de dois métodos diferentes de partição.
Na primeira, os dados são particionados \textbf{aleatoriamente} em $n$ partes.
Já para o teste da clusterização, agrupamos os mesmos dados em $n$ \emph{clusters} criados pelo algoritmo \emph{k-means}.
Cada algoritmo de compressão é executado sobre as $n$ partições (grupos), e as métricas são computadas como uma média aritmética do desempenho alcançado em cada partição.

\subsection{Partição aleatória}
A partição aleatória consiste em dividir o \emph{dataframe} original em $n$ novos \emph{dataframes}.
Para evitar possíveis \emph{bias} presentes na base original (por exemplo, se os dados já estivessem originalmente agrupados), precisamos que essa divisão seja aleatória. 

Para isso, foi utilizado o método $.sample()$, que retorna uma amostra aleatória de items. 
O parâmetro $frac=1$, faz com que o método retorne uma amostra com todos os dados (porém de maneira randômica).
Depois disso, fracionamos o array utilizando a função $array\_split()$ da biblioteca $numpy$.

\begin{lstlisting}[language=Python, caption=Partição aleatória de dados]
# Data to compress: Compress the original data raw
raw_text = df['content']

#Shuffle raw text
shuffle = raw_text.sample(frac=1)
partitions = np.array_split(shuffle, n_partitions)
\end{lstlisting}

\subsection{Clusterizacao}
Tecnologia e implementação
Mostrar parametros
mostrar gráficos

\section{Resultados}
\subsection{Metodologia}
Explicar sobre a extassao de medidas de comparacao

\subsection{Resultados pré clusterizacao}

\subsection{Resultados pós clusterizacao}

\section{Conclusao}

\section{Trabalhos relacionados e melhorias}






%\chapter{Estado da Arte}\label{cap:estArte}

\lipsum[34]

\section*{Trabalhos Relacionados a Isto}\label{sec:primTrab}
\addcontentsline{toc}{section}{Trabalhos Relacionados a Isto}

\lipsum[34-36]
%\chapter{Materiais e Métodos}\label{cap:ferramentas}

\lipsum[43-45]

\section{Considerações Finais}

\lipsum[23]


% PARTE
%\part{Proposta}
%\chapter{Sistema Proposto}\label{cap:proposta}

Esse trabalho propõe um sistema de... 


\section{Primeira Parte do Sistema Proposto}

\lipsum[67]

\section{Considerações Finais}

\lipsum[68]


% PARTE
%\part{Parte Final}
%\chapter{Resultados e Discussão}\label{cap:resultados}

\lipsum[73]

\section{Base de Dados}

\lipsum[72]

\section{Considerações Finais}

\lipsum[74]
%\chapter*{Conclusões e Trabalhos Futuros}\label{cap:conclusao}
\addcontentsline{toc}{chapter}{Conclusão e Trabalhos Futuros}

\lipsum[81]

\section*{Conclusões}

\lipsum[82-84]

\section*{Trabalhos Futuros}

\lipsum[85] 

% ----------------------------------------------------------
% ELEMENTOS PÓS-TEXTUAIS (Referências, Glossário, Apêndices)
% ----------------------------------------------------------
\postextual

\begin{thebibliography}{9} 
\bibitem{HL}HIRSCHBERG, D.S; LELEWER D.A;
\emph{Data compression, }Computing Surveys 19.3, 1987.

\bibitem{Ble}BLELLOCH G.E;
\emph{Introduction to Data Compression, } Carnegie Mellon, 2013

\bibitem{BT}BERTSEKAS D.P; TSITSIKLIS J.N;
\emph{Introduction to Probability} M.I.T, Lecture Notes Course 6.041-6.431, 2000

\bibitem{MDN}COMPRESSÃO em HTTP; 
MDN Web Docs, 2021; 
Disponível em: <\url{https://developer.mozilla.org/pt-BR/docs/Web/HTTP/Compression}>; Acesso em: 16 de dez de 2021.

\bibitem{Camb}COMPRESSION Techniques; 
Cambridge University, Raspberry Pi Foundationand National Centre for Computing education;
 Disponível em : <\url{https://isaaccomputerscience.org/concepts/data_compr_loss?examBoard=all&stage=all.}> Acesso em: 30 de maio de 2022.
 
\bibitem{MicDocs}COMPACTAÇÃO de dados; 
Microsoft Docs. 2021; 
Disponível em: <\url{https://docs.microsoft.com/pt-br/sql/relational-databases/data-compression/data-compression?view=sql-server-ver15}>; Acesso em: 16 de dez de 2021.

\bibitem{Goog}CLUSTERING in Machine Learning; Google Developers; 
Disponível em:  <\url{https://developers.google.com/machine-learning/clustering/overview}>; Acesso em: 30 de maio de 2022

\end{thebibliography}
	
% Glossário (Consulte o manual)
%\glossary

% Apêndices
% \include{postextual/apendices}

% Anexos
% \include{postextual/anexos}

% Índice remissivo (Consultar manual)
%\phantompart
%\printindex

\end{document}
